\documentclass[11pt]{article}
\usepackage{titlesec}
\usepackage{fancyhdr}
\usepackage{charter}
\usepackage{hyperref}
\usepackage{tabularx}
%\renewcommand{\familydefault}{\sfdefault}

\usepackage[document]{ragged2e}

\usepackage[usenames,dvipsnames]{xcolor}
\usepackage[margin=1in]{geometry}
\pagestyle{fancy}
\hypersetup{colorlinks, breaklinks, urlcolor=OliveGreen, linkcolor=OliveGreen}
\fancyhf{}
\rhead{\Large\scshape{Curriculum Vitae}}
\lhead{\Large\scshape{Tara N. Furstenau}}
\cfoot{\thepage}

\titleformat{\section}
	{\large\scshape\bfseries}{}{1em}{}
\titleformat{\subsection}
	{\normalsize\bfseries}{}{1em}{}
\begin{document}
%--------------------------------------------------------------------------------------------
%                                CONTACT INFORMATION
%--------------------------------------------------------------------------------------------
\section*{Contact Information}
\begin{tabular}[c]{p{3.5in}ll}
The Biodesign Institute& Phone & +1 (702) 907-8272\\
Arizona State University& Email & \href{mailto:tfursten@asu.edu}{tfursten@asu.edu}\\
PO Box 875301 & Website & \href{http://tfursten.github.io}{http://tfursten.github.io}\\
Tempe, Arizona 85287-5301&&\\
%Phone & +1 (702) 907-8272\\
%Email & \href{mailto:tfursten@asu.edu}{tfursten@asu.edu}\\
%Website & \href{http://tfursten.github.io}{http://tfursten.github.io}\\
\end{tabular}

%--------------------------------------------------------------------------------------------
%                                RESEARCH FOCUS
%--------------------------------------------------------------------------------------------
%\section*{Current Research}
%I am a computational biologist interested in studying evolution and population genetics in a spatial context. Most of my research focuses on consequences of isolation-by-distance, a phenomenon which results in the formation of spatial genetic structure in a population due to limited dispersal. I have developed several spatially-explicit individual based simulations to better understand how isolation-by-distance is affected by different dispersal distributions and different mating types. I am also working on developing a Bayesian method for estimating neighborhood size which roughly quantifies the spatial-genetic structure in a population due to isolation-by-distance.

%--------------------------------------------------------------------------------------------
%                                EDUCATION
%--------------------------------------------------------------------------------------------


\section*{Education}
\begin{tabular}[c]{lllll}
Ph.D.& Molecular and Cellular Biology& Arizona State University&2016\\
&\multicolumn{3}{l}{Advisor: Reed Cartwright}\\
B.S.&Bioinformatics and Genomics&Arizona State University& 2010\\
&\multicolumn{3}{l}{\textit{magna cum laude}}
\end{tabular}
%--------------------------------------------------------------------------------------------
%                                PUBLICATIONS
%--------------------------------------------------------------------------------------------
\section*{Publications}
%\subsection*{Submitted}
\renewcommand{\arraystretch}{1.5}
\begin{tabularx}{\linewidth}{l X}
&\textbf{Furstenau, TN}, and RA Cartwright (2016). The effect of the dispersal kernel on isolation-by-distance in a continuous population. \textit{PeerJ} 4:e1848. \href{https://doi.org/10.7717/peerj.1848}{doi:10.7717/peerj.1848}\\

&Pizzio GA, Paez-Valencia J, Khadilkar AS, Regmi K, Patron-Soberano A, Zhang S, Sanchez-Lares J, \textbf{Furstenau T}, Li J, Sanchez-Gomez C, Valencia-Mayoral P, Yadav UP, Ayre BG and RA Gaxiola (2015).\textit{ Arabidopsis} proton-pumping pyrophosphatase \textit{AVP1} expresses strongly in phloem where it is required for PPi metabolism and photosynthate partitioning. \textit{Plant Physiology} \textbf{167}:1541-1553. \href{http:/​/​dx.​doi.​org/​10.​1104/​pp.​114.​254342}{doi:10.1104/pp.114.254342}\\
\end{tabularx}
%\subsection*{Published}
%\begin{tabular}[c]{lp{6.5in}}
%--------------------------------------------------------------------------------------------
%                                PRESENTATIONS
%--------------------------------------------------------------------------------------------
\section*{Presentations}
\subsection*{Talks}
\renewcommand{\arraystretch}{1}
\begin{tabularx}{\linewidth}{l X}
\textbf{2015}&\textit{Bayesian estimation of neighborhood size using composite marginal likelihoods}\\
&\textbf{Society for Molecular Biology and Evolution} $\cdotp$ Vienna, Austria\\
2015&\textit{Bayesian estimation of neighborhood size using composite marginal likelihoods}\\
&Molecular and Cellular Biology Colloquium $\cdot$ The Biodesign Institute\\
2014&\textit{Evolution of Self-Incompatibility: Investigating the role of self-incompatibility systems in the prevention of biparental inbreeding.}\\
&Molecular and Cellular Biology Colloquium $\cdotp$ The Biodesign Institute\\
2013&\textit{Evolution of Self-Incompatibility: Investigating the role of self-incompatibility systems in the prevention of biparental inbreeding}\\
&Molecular and Cellular Biology Colloquium $\cdotp$ Arizona State University\\
2012&\textit{Is the H$^+$-pyrophosphatase involved in the regulation of sucrose transport in plants?}\\
&Molecular and Cellular Biology Colloquium $\cdotp$ Arizona State University\\
\end{tabularx}
\subsection*{Posters and Abstracts}
\begin{tabularx}{\linewidth}{l X}
2014&\textit{The effect of the dispersal distribution on isolation-by-distance in a continuous population}\\
&Society for the Study of Evolution $\cdotp$ Raleigh, NC\\
2014&\textit{Characterization of Transgenic Arabidopsis thaliana overexpressing AVP1 and PLAFP}\\
&Undergraduate Research Poster Symposium $\cdotp$ Arizona State University $\cdotp$ Tempe, AZ\\
&Presented by Sean Wilson (undergraduate mentee)\\
2012&\textit{H$^+$-PPase AVP1 is necessary for phloem development in Arabidopsis thaliana}\\
&Molecular and Cellular Biology Graduate Student Retreat $\cdotp$ Tempe, AZ\\
2012&\textit{H$^+$-PPase AVP1 is necessary for phloem development in Arabidopsis thaliana}\\
&American Society of Plant Biologists $\cdotp$ Austin, TX\\
\end{tabularx}

%--------------------------------------------------------------------------------------------
%                                RESEARCH EXPERIENCE
%--------------------------------------------------------------------------------------------
\section*{Research Experience}
\begin{tabularx}{\linewidth}{lX}
2016-Present& Postdoctoral Scholar with Viacheslav Fofanov\\
&Informatics and Computing Program\\ \vspace{2mm}
&Northern Arizona University $\cdotp$ Flagstaff, AZ\\
2013-2016&Graduate Student with Reed Cartwright\\
&Center for Human and Comparative Genomics\\
&The Biodesign Institute\\ \vspace{2mm}
&Arizona State University $\cdotp$ Tempe, AZ\\ 
2010-2013&Research Assistant with Roberto Gaxiola\\
&School of Life Science\\ \vspace{2mm}
&Arizona State University $\cdotp$ Tempe, AZ\\ 
2009-2010&Undergraduate Student Researcher with Lei Lei\\
&School of Life Science\\
&Arizona State University $\cdotp$ Tempe, AZ\\
\end{tabularx}



%--------------------------------------------------------------------------------------------
%                                TEACHING EXPERIENCE
%--------------------------------------------------------------------------------------------
\section*{Teaching Experience}
\subsection*{Arizona State University}
Courses:\\
\begin{tabularx}{\linewidth}{llX}
2014&General Genetics& Head TA\\
2014& Introduction to Computational Molecular Biology& Innovative TA\\
2013&Concepts in Plant Biology iCourse& Instructor\\
2011-2015&General Genetics&Teaching Associate\\
2011&General Biology I \& 2 Laboratory& Teaching Associate\\
2010&Genetic Engineering and Society Laboratory&Teaching Associate\\\\
\end{tabularx}
Undergraduate Mentorship:\\
\begin{tabularx}{\linewidth}{lX}
2011-2013&Honors Thesis Mentor $\cdotp$ Sean Wilson\\
&Wilson S, Furstenau T, and R Gaxiola. Characterization of Transgenic \textit{Arabidopsis thaliana} Overexpressing a Type I H\textsuperscript{+} Pyrophosphatase and the Phloem Lipid-Associated Family Protein. \href{http://hdl.handle.net/2286/R.I.23607}{http://hdl.handle.net/2286/R.I.23607}\\
\end{tabularx}


%--------------------------------------------------------------------------------------------
%                                AWARDS
%--------------------------------------------------------------------------------------------
\section*{Awards}
\begin{tabularx}{\linewidth}{llX}
2015 & ASU &Graduate Education Travel Award\\
2015 & ASU &School of Life Sciences Travel Award\\
\end{tabularx}

%--------------------------------------------------------------------------------------------
%                                SERVICE
%--------------------------------------------------------------------------------------------
\section*{Service and Outreach}
\begin{tabularx}{\linewidth}{lX}
2015 &Molecular and Cellular Biology and Microbiology Retreat Poster Judge\\
2015 &Software Carpentry Workshop Helper\\
2015 &Night of the Open Door Volunteer\\
2010-present &Ask-A-Biologist Volunteer Corespondent\\
2011-2013&Green Labs Initiative Coordinator and Promoter\\
2012-2013&Phosphorus Sustainability Research Coordination Network Core Participant\\
2011-2012&Obama Scholars Mentor\\
\end{tabularx}
%--------------------------------------------------------------------------------------------
%                                SOCIETY MEMBERSHIPS
%--------------------------------------------------------------------------------------------
\section*{Society Memberships}
Society for the Study of Evolution\\
Society for Molecular Biology and Evolution\\
Graduate Integrative Society for Environmental Interdisciplinary Research\\
Central Arizona Chapter of the Association for Women in Science
%--------------------------------------------------------------------------------------------
%                                PROFESSIONAL DEVELOPMENT
%--------------------------------------------------------------------------------------------
\section*{Professional Development}
\begin{tabularx}{\linewidth}{l X}
2013&Next Generation Population Genomics for Non-model Taxa Workshop\\
&American Genetics Association $\cdotp$ Cornell University $\cdotp$ Ithica, NY\\

2011& Univector Plasmid-Fusion System training with Kendal Hirschi\\
&Childrens Nutritional Research Center $\cdotp$ Baylor College of Medicine $\cdotp$ Houston, TX\\

\end{tabularx}
%--------------------------------------------------------------------------------------------
%                                PROGRAMMING LANGUAGES
%--------------------------------------------------------------------------------------------
\section*{Programming Languages}
\textsc{C++, python, R, bash, \LaTeX, html/css, OpenBUGS, Mathematica}
\end{document}